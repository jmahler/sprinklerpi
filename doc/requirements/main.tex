
% {{{ preamble

\documentclass[12pt,letterpaper]{article}

\usepackage{url}
\usepackage{hyperref}

\usepackage{graphicx}

\usepackage{makeidx}
\makeindex
\usepackage[nottoc]{tocbibind}
% like \index but prints the entry as well
\newcommand{\pindex}[1]{#1\index{#1}}

\usepackage{vmargin}  % make the margins a bit smaller
\setmarginsrb{1.0in}{1.0in}{1.0in}{1.0in}{0in}{0.4in}{0.0in}{0.40in}

\usepackage{setspace}

\raggedright
%\setlength{\parindent}{0.5in}
%\noindent
\usepackage{parskip}

\usepackage{sectsty}
\sectionfont{\large}

% no numbers in bibliography, good
\usepackage[backend=biber,autocite=footnote,
			bibstyle=authortitle,citestyle=verbose-inote]{biblatex}

\addbibresource{references.bib}

% redefine first level enumerate
\renewcommand*{\theenumi}{\thesection.\arabic{enumi}}
\renewcommand*{\labelenumi}{\theenumi}
% redefine second level enumerate
\renewcommand*{\theenumii}{\theenumi.\arabic{enumii}}
\renewcommand*{\labelenumii}{\theenumii}

% }}}

\begin{document}

% {{{ title page

\vspace*{1.0in}

\centerline{\Large \textbf{SprinklerPI}}
\centerline{(Requirements)}

\vspace{0.5in}

\begin{center}
\begin{tabular}{c}
Jeremiah Mahler \\
EECE 490B, CSU Chico \\
\today
\end{tabular}
\end{center}

\thispagestyle{empty}

\vfill

% }}}

\pagebreak
\tableofcontents
\addtocontents{toc}{~\hfill{Page}\par}
\pagebreak

% {{{ Project Overview
\section{Project Overview}

This project is a sprinkler timer for watering lawns and plants
on a regular schedule.
It is network accessible and can be configured through a web browser.
The system is also expandable.
It can control from one valve to several hundred valves.

% users
The primary user of this project would be a homeowner who has
lawns or other plants that need watering on a regular schedule.
Businesses with landscaping that needs watering could also use
this system.

The system will need to be installed by someone familiar with
irrigation systems (timers and valves).
Since this system is network accessible it will need to be initially
configured by someone familiar with Ethernet and Wireless networks.

After installation anyone should be capable of configuring the
water schedule through the web based interface.

% environmental impact
This project has a positive environmental impact as a result of the
more efficient use of water resources.

It is estimated that this project will take four months to complete.
It is estimated that the hardware will cost approximately \$225.

% }}}

\pagebreak

% {{{ Features and Functions
\section{Features and Functions}

\begin{enumerate}
\item Product shall be capable of driving residential grade sprinkler valves.
	\begin{enumerate}
	\item 24 volts AC control signal.
	\item 270 mA rms current.
	\item Maximum current draw should limited to no more than 300 mA.
	\item Residential, $1/2"$ to $1"$ NPT size.
	\end{enumerate}

\item Product shall be expandable to control more valves.
	\begin{enumerate}
	\item Each control group shall allow eight valves.
	\item For one control group only one valve can be on at a time.
	\item It will support a minimum of one to three control groups
		(8 to 24 valves).
	\end{enumerate}

\end{enumerate}
% }}}

% {{{ User Interface
\section{User Interface}

\begin{enumerate}
\item Web based interface for all configuration/settings.
	\begin{enumerate}
	\item Setting timer schedules.
	\item Manually starting/stopping valves.
	\item Viewing history of valve run times.
	\item User authentication.
	\end{enumerate}
\item Network configuration.
	\begin{enumerate}
	\item Configuration performed on Linux based Rasberry PI.
	\item Standard ssh terminal access.
	\end{enumerate}
\item Hardware interface.
	\begin{enumerate}
	\item LED indicating when valve is on.
	\item LED indicating when power is applied.
	\item Manual shutoff of all valves.
	\end{enumerate}
\end{enumerate}

% }}}

% {{{ External Interface Requirements
\pagebreak
\section{External Interface Requirements}
\begin{enumerate}
\item Communication between the Linux computer (Rasberry PI) and
		the control groups will be done using SPI.
\item Expansion is supported by daisy chaining SPI.
\end{enumerate}

% }}}

% {{{ Installation Requirements
\section{Installation Requirements}
\begin{enumerate}
	\item LEDs must be provided to indicate when a valve is on.
	\item LEDs should indicate when power is applied.
	\item Terminals for checking primary voltages must be available.
	\item PCB should clearly indicate voltages.
\end{enumerate}
% }}}

% {{{ Design Constraints
\section{Design Constraints}

\begin{enumerate}
	\item A Linux computer capable of communication over SPI.
	\begin{enumerate}
		\item Currently a RasberryPI\autocite{rpi} satisfies these requirements.
	\end{enumerate}
	\item All components should fit inside a standard timer box:
		$9.25"$ x $10.75"$ x $3.25"$
	\item On power up all sprinkler valves must remain off.
\end{enumerate}
% }}}

% {{{ Test Requirements
\section{Test Requirements}

\begin{enumerate}
	\item 110 volt AC terminals accessible.
	\item 24 volt AC terminals accessible.
	\item 5 volt DC terminals accessible.
	\item LEDs to indicate when a valve is on.
\end{enumerate}
% }}}

% {{{ Packaging
\section{Packaging}

\begin{enumerate}
	\item All components should mount to a backplane using standoffs.
	\item All items should fit inside a standard sprinkler timer box.
	\item All circuits should be manufactured to used PCBs.
\end{enumerate}
% }}}

% {{{ Environmental Requirements
\section{Environmental Requirements}

\begin{enumerate}
\item The product shall operate at temperatures from -40 degF
		up to 150 degF.
\item Push on connectors or ribbon cables can be used (no vibration).
\item All components must tolerate high humidity environments without
	degradation.
\end{enumerate}
% }}}

% {{{ Power Supply Requirements
\section{Power Supply Requirements}

\begin{enumerate}
	\item Power supply is from a 110 volts AC 15A circuit.
	\item Provide 5 volts DC at 600 mA.
	\begin{enumerate}
		\item Current should be limited to no more than 1 A.
	\end{enumerate}
	\item Provide 24 volts AC at 1 A.
	\begin{enumerate}
		\item Should be fused at 1 A.
	\end{enumerate}
	\item RasberryPI (Linux) requires 5 volts at 500 mA.
\end{enumerate}

% }}}

\pagebreak
\printbibliography[heading=bibintoc]

\end{document}

