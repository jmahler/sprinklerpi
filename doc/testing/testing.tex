
\documentclass{article}

\usepackage{graphviz}
\usepackage{url}
\usepackage{hyperref}
\usepackage{fullpage}
\usepackage{parskip}
\usepackage{fancyvrb}
\usepackage{amsmath}

\usepackage{listings}
\lstset{numbers=left,
		basicstyle=\footnotesize,
		captionpos=b,
		xleftmargin=0.3in}

\providecommand{\e}[1]{\ensuremath{\times 10^{#1}}}

\VerbatimFootnotes

\raggedright

% Change enumerate section numbering
\renewcommand*{\theenumii}{\theenumi.\arabic{enumii}}
\renewcommand*{\labelenumii}{\theenumi.\arabic{enumii}}

\begin{document}

% {{{ title page
\vspace*{1.0in}

\centerline{\Large \textbf{SprinklerPI}}
\centerline{(Product Test Plan)}

\vspace{0.5in}

\begin{center}
\begin{tabular}{c}
Jeremiah Mahler \\
EECE 490B, CSU Chico \\
\today
\end{tabular}
\end{center}

\thispagestyle{empty}

\vfill

\pagebreak
% }}}

\thispagestyle{empty}
\tableofcontents
\clearpage

% {{{ Information Required for Execution
\section{Information Required for Execution}

\subsection{Purpose}

The purpose of this test protocol is to verify the full and complete
operation of the SprinklerPI system.

\subsection{Scope}

This test protocol should be executed to verify that each principal feature
and function performs within specification called out by the engineering
requirements document. In addition, other necessary specifications shall
be tested such as specific and necessary user, installation and power
requirements. The testing called out in this protocol is subject exclusively
to those selected specifications provided for the SprinklerPI.

\subsection{Responsibilities}

It is the responsibility of the assigned test engineer to execute all tests
included herein to the best of their ability. If necessary, seek additional
assistance to execute tasks.

\subsection{References}

Not applicable at this time.

%\subsection{Definitions}

\subsection{Equipment/Supplies}

\begin{itemize}
\item 110 volts AC power outlet.
\item Digital volt meter.
\item One Sprinkler valve, residential 270 mA.
\end{itemize}

\subsection{Precautions \& Warnings}

This test plan contains certain warnings and cautions designed to alert the
test engineer while performing tests. The following illustrates each of
these messages and how to recognize them.

\fbox{
\textbf{WARNING: \textless message\textgreater}
}

The ``WARNING: Message'' alerts the user about safety issues that are of
the highest importance, such as possible injury to the operator.

\fbox{
\textbf{CAUTION: \textless message\textgreater}
}

The ``CAUTION: Message'' alerts the user to issues concerning possible
damage to the equipment or that can lead to erroneous test results.

% }}}

\section{Testing Features and Functions}

% {{{ Power Supply
\subsection{Power Supply}

\begin{enumerate}
\item Equipment Required
	\begin{itemize}
	\item 110 volts AC, standard U.S. residential output.
	\item Digital multi meter.
	\end{itemize}
\item Input
	\begin{itemize}
	\item 110 volts AC
	\end{itemize}
\item Output
	\begin{itemize}
	\item 24 volts AC
	\item 5 volts DC
	\item Pass: 24$\pm1.2$ volts AC measured (5\%).
		5$\pm0.15$ volts DC measured (3\%).
	\item Fail: voltages measured out of range.
	\end{itemize}
\item Test Description \\

With the 110 volts AC applied the voltage supply should
produce both a 25 volts AC output and a 5 volts DC output.
These values are measured using a digital multi meter.
The test passes if they are within the tolerance range,
otherwise the test fails.

\item Test Results \\
	\vspace{0.2in}
	\begin{tabular}{|l|l|l|l|}
		\hline
		& Value & Pass/Fail & Results/Data\hspace{2in} \\
		\hline
		24 volts AC &&& \\
		\hline
		5 volts DC &&& \\
		\hline
	\end{tabular}
\end{enumerate}
% }}}

% {{{ Control
\subsection{Control}

\begin{enumerate}
\item Equipment Required
	\begin{itemize}
	\item 110 volts AC, standard U.S. residential output
	\end{itemize}
\item Input
	\begin{itemize}
	\item Input command, through user interface, to turn valve on.
	\end{itemize}
\item Output
	\begin{itemize}
	\item LED indicator for valve circuit when sprinkler valve on.
	\item Pass: LED goes on less than one second after on command.
		LED goes off less than one second after off command.
	\item Fail: LED does not respond.  LED for wrong circuit responds.
	\end{itemize}
\item Test Description \\

With the full system powered on a command signal
given through the user interface should turn
on the corresponding driver circuit which should illuminate
the corresponding LED indicator.
This process should be repeated for each of the
eight circuits of the control/driver.

\item Test Results \\
	\vspace{0.2in}
	\begin{tabular}{|l|l|l|}
		\hline
		& Pass/Fail & Results/Data\hspace{2in} \\
		\hline
		1 && \\
		\hline
		2 && \\
		\hline
		3 && \\
		\hline
		4 && \\
		\hline
		5 && \\
		\hline
		6 && \\
		\hline
		7 && \\
		\hline
		8 && \\
		\hline
	\end{tabular}
\end{enumerate}
% }}}

% {{{ Driver
\subsection{Driver}

\begin{enumerate}
\item Equipment Required
	\begin{itemize}
	\item 110 volts AC, standard U.S. residential output
	\item Sprinkler valve, 270 mA residential
	\end{itemize}
\item Input
	\begin{itemize}
	\item Input command, through user interface, to turn valve on.
	\end{itemize}
\item Output
	\begin{itemize}
	\item Detected operation of sprinkler valve.
	\item Pass: Valve switches on less than one second after command.
		Valve switches off less than one second after command.
	\item Fail: No response or slow response of valve.
	\end{itemize}
\item Test Description \\

With the full system powered on a command signal
given through the user interface should turn
the corresponding sprinkler valve on.
This process should be repeated for each of the
eight circuits of the control/driver.

\item Test Results \\
	\vspace{0.2in}
	\begin{tabular}{|l|l|l|}
		\hline
		Valve & Pass/Fail & Results/Data\hspace{2in} \\
		\hline
		1 && \\
		\hline
		2 && \\
		\hline
		3 && \\
		\hline
		4 && \\
		\hline
		5 && \\
		\hline
		6 && \\
		\hline
		7 && \\
		\hline
		8 && \\
		\hline
	\end{tabular}
\end{enumerate}
% }}}

\section{Other Testing}

% {{{ User Interface
\subsection{User Interface}

\begin{enumerate}
\item Equipment Required
	\begin{itemize}
	\item Computer with web browser and network connection.
	\end{itemize}
\item Input
	\begin{itemize}
	\item Turn valve on/off.
	\item Check status of valve.
	\item Schedule time and duration to run valve.
	\end{itemize}
\item Output
	\begin{itemize}
	\item Valves go on when commanded.
	\item Status of operating valve shown when on.
	\item Scheduling a time and duration works as expected.
	\item Pass: Valves operate according to the commands.
	\item Fail: Response is not expected.
	\end{itemize}
\item Test Description \\

The basic commands given through the web interface are:
on/off valve, show current status, schedule a time/duration.
These tests verify these operations.

\item Test Results \\
	\vspace{0.25in}
	\begin{tabular}{|l|l|l|l|l|l|c|}
		\hline
		\multicolumn{7}{|c|}{Valves On/Off} \\
		\hline
		\# & On? & On Status? & Off? & Off Status? & Pass/Fail & \hspace{0.7in}Notes\hspace{0.7in} \\
		\hline
		1 &&&&&& \\
		\hline
		2 &&&&&& \\
		\hline
		3 &&&&&& \\
		\hline
		4 &&&&&& \\
		\hline
		5 &&&&&& \\
		\hline
		6 &&&&&& \\
		\hline
		7 &&&&&& \\
		\hline
		8 &&&&&& \\
		\hline
	\end{tabular}

	\vspace{0.25in}
	\begin{tabular}{|l|l|l|l|l|}
		\hline
		\multicolumn{5}{|c|}{Schedule} \\
		\hline
		\# & Time Start & Time End & Pass/Fail & \hspace{0.5in}Notes\hspace{0.5in} \\
		\hline
		& now & now + 1 minute & & \\
		\hline
		& now & now + 5 minute & & \\
		\hline
	\end{tabular}

\end{enumerate}

% }}}

\end{document}

